Im Zuge des Praktikums haben wir sowohl positive als auch negative Erfahrungen
gemacht. Diese legen wir im Folgenden dar.

\subsection{Positive Erfahrungen}
Grundsätzlich gesteltate sich die Zusammenarbeit mit den anderen Beteiligten
des UH Projekts als äußerst locker unde einfach. Durch die ständige Anwesenheit
von Personen im IRC Channel und die Aufgeschlossenheit, Hilfsbereitschaft und
Geduld der Entwickler des UH Teams wurde unsere Arbeit deutlich erleichtert,
sei es beim Verstehen komplizierter Zusammenhänge und Vorgänge oder beim
finden von Fehlern in unserem Code.

Durch die regelmäßigen sonntäglichen Treffen wurde eine einfache
Kommunikatiosplattform geschaffen, über die wir unsere Ergebnisse kommunizieren
und Feedback aus der Entwicklergemeinde einholen konnten. Dies führte dazu,
dass unser Projekt innerhalb eines akzeptablen Rahmens blieb und gleichzeitig
vom UH Team ohne große Probleme übernommen wurde.

Ein weiterer Punkt, der hierzu beigetragen hat ist die Tatsache, dass der
Editor Code vom restlichen Code des Spiels in großen Teilen unabhängig ist.
Hierdurch waren wir auch durch große Änderungen am Spiel selbst nur
mäßig betroffen und konnten unser Projekt unabhängig von der Entwicklung
des Spiels selbst gestalten. Daraus resultierend kam es während der gesamten
Entwicklung zu keinem einzigen Merge-Konflikt
% TODO Moritz: hattest du auch keinen?
und die einzigen Änderungen am Spiel, die Auswirkungen auf unseren Code hatten,
waren groß angelegte Refactorings der Architektur.

\subsection{Negative Erfahrungen}
- immer wieder umstellen des map-formates (dreimal!)
- für zwei projekte gearbeitet: fife und uh => mehr aufwand

\subsection{Schlüsse}
- Bessere koordination durch mehr aktives nachfragen
- vorher feststellen ob sich relevante projektteile demnächst ändern
- klare abgrenzung hilft (scope, code)
- relevanz und bedürfnis nach projekt wichtig

