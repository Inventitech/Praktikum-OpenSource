Im Zuge des Praktikums haben wir sowohl positive als auch negative Erfahrungen
gemacht. Diese legen wir im Folgenden dar.

\subsection{Positive Erfahrungen}
Grundsätzlich gesteltate sich die Zusammenarbeit mit den anderen Beteiligten
des UH Projekts als äußerst locker und einfach. Durch die ständige Anwesenheit
von Personen im IRC Channel und die Aufgeschlossenheit, Hilfsbereitschaft und
Geduld der Entwickler des UH Teams wurde unsere Arbeit deutlich erleichtert,
sei es beim Verstehen komplizierter Zusammenhänge und Vorgänge oder beim
finden von Fehlern in unserem Code.

Durch die regelmäßigen sonntäglichen Treffen wurde eine einfache
Kommunikatiosplattform geschaffen, über die wir unsere Ergebnisse kommunizieren
und Feedback aus der Entwicklergemeinde einholen konnten. Dies führte dazu,
dass unser Projekt innerhalb eines akzeptablen Rahmens blieb und gleichzeitig
vom UH Team ohne große Probleme übernommen wurde.

Ein weiterer Punkt, der hierzu beigetragen hat ist die Tatsache, dass der Editor
Code vom restlichen Code des Spiels in großen Teilen unabhängig ist.
Hierdurch waren wir auch durch große Änderungen am Spiel selbst nur mäßig
betroffen und konnten unser Projekt unabhängig von der Entwicklung des Spiels
selbst gestalten. Daraus resultierend kam es während der gesamten Entwicklung zu
keinem einzigen Merge-Konflikt und die einzigen Änderungen am Spiel, die
Auswirkungen auf unseren Code hatten, waren groß angelegte Refactorings der
Architektur. Hier war der Zeitpunkt für die Editorerstellung ungünstig.

\subsection{Negative Erfahrungen}
Es gab jedoch auch Probleme bei der Umsetzung unseres Vorhabens.
So gingen die oben genannten Änderungen an der Spielarchitektur nicht spurlos
an uns vorbei. Es waren einige Anpassungen nötig, wie in den vorangegangenen Kapiteln
beschrieben, um unseren Editor-Code wieder funktionstüchtig zu machen.
Diese Umstellung der Architektur wurden uns leider nicht ausführlich genug
kommuniziert, wahrscheinlich auch deshalb, weil keinem der Entwickler klar war,
dass dies auch Auswirkungen auf unseren Code haben könnte.

Des weiteren änderte sich einmal während der Entwicklungszeit das Format der
Maps geringfügig, wodurch einige Anpassungen in unserem Loader und Saver vonnöten
waren. Auch hier wurde uns vorher keine Mitteilung gemacht.

Schließlich gab es noch einige kleinere Stolpersteine, die dadurch entstanden,
dass wir nicht nur für UH gearbeitet haben, sondern teilweise auch Code
für das FIFE Projekt geschrieben haben, da wir deren Editor verwendeten.
Dies bedingte, dass wir doppelten Koordinationsaufwand hatten: Zum einen mit
den UH Entwicklern und zum andern mit dem FIFE Team.

Schlussendlich wurde durch ein FIFE-Update das Rückgeben von Koordinatenwerten
im Editor ungenau: Statt $[45^\circ, 135^\circ, 225^\circ, 315^\circ]$ wurden
nun Werte wie $44^\circ, 46^\circ$ etc. von der {\tt getRotation()}-Methode des
Editors zurückgegeben.
Da UH grundsätzlich gegen den FIFE trunk kompiliert wird, mussten wir hierfür
zusätzlich eine Korrekturfunktion schreiben. Allein das Debugging nahm Stunden
in Anspruch.

\subsection{Schlüsse}
Aus den gemachten Erfahrungen konnten wir einige Lehren ziehen, die nicht nur
für uns selbst, sondern auch für zukünftige Teams des Open-Source-Praktikums
von Interesse sind.

\subsubsection{Aktives Nachfragen}
Um die Koordination mit den Entwicklern des Open Source Projekts zu verbessern,
hilft es, eigenständig Nachfragen anzustellen. Fragen wurden in
unserem Projekt stets mit Hilfsbereitschaft aufgenommen. Hierdurch ist es möglich,
Probleme schneller zu lösen. Auch können so zukünftige Konflikte vermieden werden,
etwa wenn größere Änderungen geplant sind. Da die Entwickler oft keinen guten
Einblick in das bearbeitete Thema haben, ist ihnen oft nicht klar, ob ihre eigenen
Entscheidungen Konsequenzen für das Projekt haben. Eine kurze Nachfrage, ob die
geplante Änderung relevante Codeteile betrifft hilft, einfach Klarheit zu schaffen
und Vorbereitungen für die anstehenden Änderungen zu treffen.

\subsubsection{Klare Abgrenzung}
Eine klare Abgrenzung des geplanten Features vom Rest des Open Source Projekts
erleichtert eine Koordination mit anderen Entwicklungsmaßnahmen und verhindert
Konflikte und Integrationsprobleme. In unserem Fall war nicht nur der Code
größtenteils unabhängig vom Hauptprojekt, er wurde sogar in einer eigenen
Verzeichnisstruktur abgelegt.

\subsubsection{Bedürfnis erfüllen}
Das wichtigste bei der Auswahl eines guten Projektes ist die Relevanz des
bearbeiteten Themas und das Bedürfnis des Projekts für das erstellte Feature.
In unserem Fall wird der Map-Editor dringend gebraucht. Dies bedeutet, dass
unsere Implementierung auf jeden Fall angenommen wird, auch wenn noch Defizite
in der Umsetzung bestehen.

