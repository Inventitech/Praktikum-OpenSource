Ein integraler Bestandteil eines jeden Spiels ist die Möglichkeit, effizient die Spielinhalte
editieren zu können. Hierzu wird in vielen Spielen ein sogenannter Mapeditor verwendet.
Dieser ermöglicht es, die Dateien zu editieren, die die Spielinhalte enthalten, wie etwa
die Landschaft, Gebäude, Akteure und vieles mehr.
Ohne ein solches Werkzeug ist das erstellen von Spielinhalten mühsam oder gar
völlig unmöglich.

\enquote{Unknown Horizons} (UH) ist ein Open Source
Aufbaustrategiespiel nach dem Vorbild der Anno-Serie von Sunflowers.
Es ist
besonders interessant, da es nur wenige Open Source Spiele mit einer aktiven
Community und einem solchen Reifengrad gibt. Als Folge konnte UH drei Teilnehmer des
Google Summer of Code dieses Jahres für sich gewinnen.

Auch für Unknown Horizons ist ein Mapeditor von besonderer Wichtigkeit. Zwar kann das
Spiel zufällige Karten generieren, jedoch benötigen die Entwickler für wiederkehrende
Tests, als auch für statische Spielelemente, wie z.B. einen Kampangenmodus oder vorgegebene
Karten mit einem festgelegten Schwierigkeitsgrad.

Daher benötigt auch UH einen solchen Editor. Im Folgenden beschreiben wir unser Vorgehen
bei der Planung und Entwicklung eines UH Mapeditors. Hierzu geben wir zunächst einen
detaillierteren Überblick über UH und das Entwicklungsteam, das hinter dem Projekt steht.
Anschließend beschreiben wir die softwaretechnischen Aspekte der Implementierung, gefolgt
von einer Beschreibung unseres Vorgehens bei der Umsetzung der Software. Abschließend
geben wir unsere persönlichen Erfahrungen und Eindrücke zum Ablauf und Ergebnis des Projekts
wieder.

