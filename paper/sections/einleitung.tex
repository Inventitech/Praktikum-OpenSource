Ein integraler Bestandteil eines jeden größeren Computerspiels ist die
Möglichkeit, effizient die Spielinhalte editieren zu können. Hierzu wird in vielen Spielen ein sogenannter Mapeditor verwendet.
Dieser ermöglicht es, die Dateien zu editieren, die die Spielinhalte enthalten, wie etwa
die Landschaft, Gebäude, Akteure und vieles mehr.
Ohne ein solches Werkzeug ist das gezielte Erstellen von Spielinhalten mühsam
oder gar völlig unmöglich.

\enquote{Unknown Horizons} (UH) \cite{uh} ist ein Open Source
Aufbaustrategiespiel nach dem Vorbild der Anno-Serie von Sunflowers.
Es ist
besonders interessant, da es nur wenige Open Source Spiele mit einer aktiven
Community und einem solchen Reifengrad gibt. Als Folge konnte UH drei Teilnehmer des
Google Summer of Code des Jahres 2011 für sich gewinnen.

Auch für Unknown Horizons ist ein Mapeditor im aktuellen Stadium von besonderer
Wichtigkeit.
Zwar kann das Spiel zufällige Karten generieren, dies genügt jedoch den
Ansprüchen des Spiels nicht mehr. Einerseits benötigen die Entwickler für
wiederkehrende Tests vorgegebene Karten mit bestimmten Eigenschaften.
Andererseits ist ein Editor für das gezielte Erstellen von statischen
Spielelementen, wie z.B. der kürzlich integrierte
Kampangenmodus, unabdingbar. Auch können nur so vorgegebene Karten mit einem
festgelegten Schwierigkeitsgrad erzeugt werden.

Im Folgenden beschreiben wir unser Vorgehen bei der Planung und Entwicklung
eines Mapeditors für UH. Hierzu geben wir zunächst einen detaillierteren
Überblick über UH und das Entwicklungsteam, das hinter dem Projekt steht.
Anschließend beschreiben wir die softwaretechnischen Aspekte der Implementierung, gefolgt
von einer Beschreibung unseres Vorgehens bei der Umsetzung der Software. Abschließend
geben wir unsere persönlichen Erfahrungen und Eindrücke zum Ablauf und Ergebnis des Projekts
wieder.

