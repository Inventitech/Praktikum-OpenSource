
\subsection{Eigene Beiträge}
Momentan ist es in UH nur möglich, Karten zufällig zu generieren. Da eine
Kampagnenfunktion geplant ist (geplant für das 2012.2 Release Mitte 2012, siehe
Roadmap [1]), benötigt UH einen Karten Editor. FIFE bietet bereits einen solchen Editor.
Dieser kann jedoch momentan nicht für UH eingesetzt werden, da UH ein anderes Format
zum Speichern der Karten benutzt. Zudem ist er von einem Usability-Standpunkt aus
verbesserungswürdig.
Um das UH-Kartenformat im Editor benutzen zu können, muss dieser ein Plugin-System
anbieten, mit dem neue Kartenformate gelesen und gespeichert werden können. Diese
Plugin-Infrastruktur existiert momentan nur für das Laden, für das Speichern fehlt sie
noch. Der Code für die Lade-Infrastruktur kann hierbei als Vorlage genutzt werden, so
dass dies nicht den Hauptteil der Arbeit einnimmt.
Zusätzlich benötigt der FIFE Editor eine Funktion, um einen zusätzlichen Plugin-Ordner
bestimmen zu können. Momentan lädt der Editor nur Plugins aus einem fest kodierten
Standardordner. Dies ist allerdings für die Integration mit UH nicht ausreichend. Dies ist
eine kleine Erweiterung des bestehenden Editor-Codes.
Sobald die Infrastruktur des Editors implementiert ist, müssen entsprechende Plugins für
das UH-Kartenformat geschrieben werden. Dies umfasst drei wesentliche Teile:
1. Es muss ein Plugin erstellt werden, das die von UH bereitgestellten Objekte
(Bodentiles und Gebäude) laden kann, um sie im Editor bereitzustellen
2. Es muss ein Plugin erstellt werden, das bestehende UH Maps laden kann
3. Es muss ein Plugin erstellt werden, das im Editor erstellte oder durch den Editor
veränderte Karten im UH Map Format speichern kann
Dies wird die Hauptarbeit des Projekts sein, da hierzu zum einen die Funktionsweise vieler
UH-Klassen sowie das Map Format verstanden, als auch der Großteil des Codes von Grund
auf neu geschrieben werden muss. Zum anderen muss die Funktionsweise von FIFE-
Plugins verstanden werden.
Wir steuern daher Folgendes zu den Projekten bei:

FIFE
• Plugin-Infrastruktur für Speichern erstellen, um Speicherfunktionalität für
verschiedene Formate implementieren zu können
• Kommandozeilen-Parameter, um Plugins aus einem zusätzlichen Ordner laden zu
können

UH
• Plugin zum Laden von UH Objekten
• Plugin zum Laden von UH Karten
• Plugin zum Speichern von UH Karten



