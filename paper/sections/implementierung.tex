Momentan ist es in UH nur möglich, Karten zufällig zu generieren. Da eine
Kampagnenfunktion geplant ist \footnote{geplant für das 2012.2 Release Mitte 2012, vgl.
\cite{roadmap}}, benötigt UH einen Karten Editor. FIFE bietet bereits einen solchen Editor.
Dieser kann jedoch momentan nicht für UH eingesetzt werden, da UH ein anderes Format
zum Speichern der Karten benutzt.

Das Ziel des Projekts war es, einen funktionierenden Mapeditor mit Basisfunktionalität
zum Laden, Editieren und Speichern von UH Maps zu implementieren. Hierfür wurde
der bestehende FIFE Editor als Grundlage hergenommen und erweitert, so dass er das
UH Mapformat verwenden kann.

Im Folgenden geben wir einen Überblick über die im Rahmen dieses Projekts erstellten
Funktionalitäten und gehen anschließend detailliert auf jede einzeln ein.

\subsection{Überblick}
Um das UH-Kartenformat im Editor benutzen zu können, muss dieser ein Plugin-System
anbieten, mit dem neue Kartenformate gelesen und gespeichert werden können. Diese
Plugin-Infrastruktur existiert momentan nur für das Laden, für das Speichern fehlt sie
noch. Der Code für die Lade-Infrastruktur kann hierbei als Vorlage genutzt werden, so
dass dies nicht den Hauptteil der Arbeit einnimmt.

Zusätzlich benötigt der FIFE Editor eine Funktion, um einen zusätzlichen Plugin-Ordner
bestimmen zu können. Momentan lädt der Editor nur Plugins aus einem fest kodierten
Standardordner. Dies ist allerdings für die Integration mit UH nicht ausreichend. Dies ist
eine kleine Erweiterung des bestehenden Editor-Codes.

Sobald die Infrastruktur des Editors implementiert ist, müssen entsprechende Plugins für
das UH-Kartenformat geschrieben werden. Dies umfasst drei wesentliche Teile:

\begin{enumerate}
\item Es muss ein Plugin erstellt werden, das die von UH bereitgestellten Objekte
(Bodentiles und Gebäude) laden kann, um sie im Editor bereitzustellen
\item Es muss ein Plugin erstellt werden, das bestehende UH Maps laden kann
\item Es muss ein Plugin erstellt werden, das im Editor erstellte oder durch den Editor
veränderte Karten im UH Map Format speichern kann
\end{enumerate}

Dies war die Hauptarbeit des Projekts, da hierzu zum einen die Funktionsweise vieler
UH-Klassen sowie das Map Format verstanden, als auch der Großteil des Codes von Grund
auf neu geschrieben werden muss. Zum anderen muss die Funktionsweise von FIFE-Plugins
verstanden werden.

\subsection{Der FIFE Editor}
% TODO editor plugins müssen hier beschrieben werden!

\subsection{Integration in die UH Codebase}
%TODO

\subsection{Kommandozeilen-Parameter}

\subsection{Plugin-Infrastruktur}
Der FIFE Editor ist bereits ein vollwertiger Mapeditor, allerdings für ein spezielles XML Format,
das für FIFE entwickelt wurde. UH hingegen verwendet SQLite Datenbanken, um Maps und Spielstände
abzuspeichern. Um den FIFE Editor folglich für UH verwenden zu können, muss dieser um eine
Funktionalität ergänzt werden, die es ihm erlaubt, diese Datenbanken korrekt zu lesen und zu schreiben.

Dieser Use-Case ist im FIFE Editor bereits vorgesehen. Hierzu wurde eine Infrastruktur geschaffen, die
es einem Editor-Plugin erlauben, einen eigenen MapLoader zu registrieren. Dieser übernimmt die
Aufgabe, Maps eines bestimmten Dateityps zu laden. Betrachtet man den bestehenden Code, so wird klar,
dass dies auch für MapSavers vorgesehen war, also Klassen, die editierte Maps in einem fremden Format
speichern können. Jedoch wurde diese Infrastruktur nicht vollständig implementiert und so ist es
nicht möglich, einen eigenen MapSaver zu registrieren.

Folglich musste zunächst einmal die MapSaver Infrastruktur an die der MapLoader angeglichen und vollständig
implementiert werden.

\subsection{Plugin zum Laden von UH Objekten}
\subsection{Plugin zum Laden von UH Karten}
\subsection{Plugin zum Speichern von UH Karten}



