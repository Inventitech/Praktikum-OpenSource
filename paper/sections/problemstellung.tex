Vor Beginn dieses Praktikums war es in UH nur möglich, Karten zufällig zu
generieren.
Da eine Kampagnenfunktion geplant ist \footnote{geplant für das 2012.2 Release Mitte 2012, vgl.
\cite{roadmap}}, benötigt UH einen Karten Editor. FIFE bietet bereits einen
solchen generischen Editor.
Dieser konnte jedoch zunächst nicht für UH eingesetzt werden, da UH ein
anderes Format zum Speichern der Karten benutzt.

Das Ziel des Projekts war es, einen funktionierenden Mapeditor mit Basisfunktionalität
zum Laden, Editieren und Speichern von UH Maps zu implementieren. Hierfür wurde
der bestehende FIFE Editor als Grundlage hergenommen und erweitert, so dass er das
UH Mapformat verwenden kann.

Im Folgenden geben wir einen Überblick über die im Rahmen dieses Projekts erstellten
Funktionalitäten und gehen anschließend detailliert auf jede einzeln ein.

\subsection{Überblick}
Es gibt einige grundsätzliche Anforderungen an einen Mapeditor, die dieser in jedem
Fall erfüllen muss:

\begin{itemize}
\item Es muss möglich sein, bestehende Karten zu laden.
\item Es muss möglich sein, beliebige Objekte des Spiels auf der Karte zu
platzieren.
\item Es muss möglich sein, eine geänderte oder neu erstellte Karte zu
speichern.
\item Der Editor muss in die bestehende Infrastruktur des UH Projekts integriert
werden.
\item Es muss möglich sein, bereits auf der Karte platzierte Objekte zu verändern oder
zu entfernen.
\end{itemize}

\subsection{Der FIFE Editor}
Die FIFE Graphikengine besitzt bereits einen eigenen Mapeditor, der in Python geschrieben
wurde. Dieser kann Maps unter
Verwendung der Engine darstellen, kreieren und verändern. Er bietet grundsätzliche
Funktionalitäten wie das Setzen, Löschen und Bearbeiten von Objekten auf der
Map, die Organisation von Objekten in verschiedene Ebenen und das Setzen von Mapeinstellungen.

Zusätzlich kann der Editor um weitere Funktionalität erweitert werden. Dies geschieht,
indem man Python Plugins schreibt, die vom Editor beim Starten dynamisch geladen
werden. Dieser Mechanismus wurde von uns genutzt, um den bestehenden Editor-Kern
um UH-spezifische Funktionen zu erweitern.

\subsection{Probleme}
Im Folgenden werden die oben beschriebenen Probleme in Unterprobleme unterteilt
und detailliert beschrieben.

\subsubsection{Integration in die UH Codebase}
Der neue Editor muss in die bestehende Infrastruktur des UH Projekts integriert werden.
In früheren Versionen wurde der gesamte Code des FIFE Editors in das UH Repository
in ein Unterverzeichnis kopiert. Mit Hilfe der Plugin-Funktionalität der neuen Versionen
des Editors ist dies jedoch nicht mehr nötig. Da eine Duplizierung des Editor-Codes
beträchtliche Folgen für die Wartbarkeit und einen nicht zu unterschätzenden Aufwand
bei zukünftigen Aktualisierungen oder Bugfixes des Editors bedeutet, werden im UH
Repository nur die UH-spezifischen Code-Teile gehostet. Dies bedeutet, dass ein
zusätzliches Skript erstellt werden musste, das den Editor aus einem UH Unterverzeichnis
heraus starten kann und die UH Plugins in diesen integriert.
Durch diese Art der Integration in die bestehende Codebase werden unnötige
Code Duplizierungen minimiert.

\subsubsection{Kommandozeilen-Parameter}
Um ein Skript wie oben beschrieben erstellen zu können, ist es vonnöten, dass
der FIFE Editor um einen zusätzlichen Kommandozeilen-Parameter erweitert wird,
der es erlaubt, ein Verzeichnis anzugeben, aus dem zur Laufzeit Plugins geladen
werden.
Standardmäßig liest der FIFE Editor nur Plugins aus seinen eigenen
Verzeichnissen ein, die UH Plugins liegen jedoch im UH Repository getrennt von
diesen.

\subsubsection{Plugin-Infrastruktur}
\label{plugininfrastruktur}
Der FIFE Editor ist bereits ein vollwertiger Mapeditor, allerdings für ein spezielles XML Format,
das für FIFE entwickelt wurde. UH hingegen verwendet SQLite Datenbanken, um Maps und Spielstände
abzuspeichern. Um den FIFE Editor folglich für UH verwenden zu können, muss dieser um eine
Funktionalität ergänzt werden, die es ihm erlaubt, diese Datenbanken korrekt zu lesen und zu schreiben.

Dieser Use-Case ist im FIFE Editor bereits vorgesehen. Hierzu wurde eine Infrastruktur geschaffen, die
es einem Editor-Plugin erlaubt, einen eigenen MapLoader zu registrieren. Dieser übernimmt die
Aufgabe, Maps eines bestimmten Dateityps zu laden. Diese Struktur sollte analog
auch für MapSavers vorhanden sein, also Plugins, die editierte Maps in einem
fremden Format speichern können. Jedoch wurde die Infrastruktur für
MapSavers -- im Gegensatz zu der Loaders-Struktur -- nicht
implementiert und so war es nicht möglich, einen eigenen MapSaver zu
registrieren.

Folglich mussten wir zunächst die MapSaver Infrastruktur bereitstellten. Wir
entwickelten sie analog zur MapLoader-Struktur.

\subsubsection{Plugin zum Laden von UH Objekten}
Damit der User mit dem Editor neue Objekte auf einer Map platzieren kann (z.B.
Gebäude oder Bodentiles), muss der Editor alle in UH verfügbaren Objekte kennen.
Hierzu wird ein Plugin benötigt, das diese Objekte beim Editorstart lädt und
diesem bereitstellt. \\
{\tt UHObjectLoader.py}

\subsubsection{Plugin zum Laden von UH Karten}
Zusätzlich zu diesem Plugin ist ein weiteres erforderlich, das den oben erwähnten MapLoader
zur Verfügung stellt, der die SQLite Datenbanken einlesen kann, in denen die Daten der
UH Maps gespeichert sind. Wenn diese Daten eingelesen wurden, müssen mit ihnen FIFE Objekte
erzeugt werden, die der Editor darstellen und bearbeiten kann. \\
{\tt UHMapLoader.py}


\subsubsection{Plugin zum Speichern von UH Karten}
Um diesen Prozess umzukehren muss ein Plugin erstellt werden, das die FIFE Objekte aus der
Engine auslesen und sie in Daten des UH Map Formats umwandeln kann. Anschließend müssen diese
Daten in eine SQLite Datenbank geschrieben werden. \\
{\tt UHMapSaver.py}

