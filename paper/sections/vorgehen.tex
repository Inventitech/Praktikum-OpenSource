Zu Beginn des Projektes haben wir zunächst mit den UH-Entwicklern gesprochen,
um festzustellen, welche Anforderungen der Editor erfüllen sollte. Hierzu
haben wir uns informell mit einigen der Entwicklern im IRC unterhalten und
festgestellt, dass der bestehende FIFE-Editor eine gute Basis für den Map
Editor darstellt.

Anschließend haben wir uns in den bestehenden Code des Editors und den Ladeprozess
von UH eingelesen, um herauszufinden, welche Schwierigkeiten bei der Implementierung
bestehen könnten und wie eine funktionstüchtige Architektur aussehen könnte.
In einer kurzen Diskussion im FIFE IRC-Channel haben wir herausgefunden, dass
derzeit am FIFE-Editor gearbeitet wird, um die Integration von anderen Kartenformaten
zu ermöglichen, dass dieser Prozess jedoch noch nicht abgeschlossen ist.

Nach gründlicher Überlegung haben wir damit begonnen, grundsätzliche Probleme aus
dem Weg zu räumen, bevor wir mit der eigentlichen Arbeit -- dem Implementieren
der Loader und Saver Plugins -- beginnen konnten. Hierzu waren einige Patches für
FIFE vonnöten, die z.B. Bugs in der Plugin-Infrastruktur und im Dateiauswahldialog
des Editors beheben. Diese Patches haben wir anschließend in den FIFE Bugtracker
gestellt, von wo sie von einem der FIFE Entwickler gemerged wurden.

Nachdem diese grundsätzlichen Probleme behoben waren, begannen wir mit der Entwicklung
eines Loader Plugins. Hierzu teilten wir die Arbeit auf und konzentrierten uns zunächst
auf das Laden der UH Objekte. Um dies zu bewerkstelligen war eine weitere intensive Beschäftigung
mit dem FIFE Editor Code und dem UH Kartenformat nötig, um weitere Implementierungsdetails
klarzustellen.

Nachdem die Objekte im Editor geladen werden konnten, präsentierten wir unseren Fortschritt
in einem der wöchentlichen Meetings von UH.

Da zu dieser Zeit gerade größere Umbauarbeiten an der Architektur des Spiels vorgenommen wurden,
kam es zu einem Konflikt mit unserem Editor-Code. Änderungen am Spiel zwangen uns, einen Teil
des Codes neu zu schreiben, um mit den neuen Features des Spiels kompatibel zu bleiben. Dies
hätte allerdings nicht vermieden werden können, da die Änderungen bereits vor dem Beginn des
Editor-Projekts geplant waren und nicht verschoben werden konnten. Da wir bereits frühzeitig
auf einem weiter fortgeschrittenen Branch des Git-Repositories gearbeitet hatten, war der
Umfang der nötigen Änderungen noch angemessen.

- Loader split
- Plugin Infrastruktur
- Saver bauen
- Rotation fixen
- Umbau wegen components
- Island saver
- Fixes vor merge
- Merge
