Während des Projekts arbeiteten wir meist separat. Die gesamte anstehende
Arbeit wurde in mehrere Unteraufgaben aufgeteilt, die wir zumeist einzeln
erledigten. Wir trafen uns jedoch einmal die Woche, um unseren Fortschritt festzustellen
und gemeinsam am Projekt weiterzuarbeiten.

% TODO fix time for all sentences here!

Zu Beginn des Projektes haben wir zunächst mit den UH-Entwicklern gesprochen,
um festzustellen, welche Anforderungen der Editor erfüllen sollte. Hierzu
haben wir uns informell mit einigen der Entwicklern im IRC unterhalten und
festgestellt, dass der bestehende FIFE-Editor eine gute Basis für den Map
Editor darstellt.

Anschließend haben wir uns in den bestehenden Code des Editors und den Ladeprozess
von UH eingelesen, um herauszufinden, welche Schwierigkeiten bei der Implementierung
bestehen könnten und wie eine funktionstüchtige Architektur aussehen könnte.
In einer kurzen Diskussion im FIFE IRC-Channel haben wir herausgefunden, dass
derzeit am FIFE-Editor gearbeitet wird, um die Integration von anderen Kartenformaten
zu ermöglichen, dass dieser Prozess jedoch noch nicht abgeschlossen ist.

Nach gründlicher Überlegung haben wir damit begonnen, grundsätzliche Probleme aus
dem Weg zu räumen, bevor wir mit der eigentlichen Arbeit -- dem Implementieren
der Loader und Saver Plugins -- beginnen konnten. Hierzu waren einige Patches für
FIFE vonnöten, die z.B. Bugs in der Plugin-Infrastruktur und im Dateiauswahldialog
des Editors beheben. Diese Patches haben wir anschließend in den FIFE Bugtracker
gestellt, von wo sie von einem der FIFE Entwickler gemerged wurden.

Nachdem diese grundsätzlichen Probleme behoben waren, begannen wir mit der Entwicklung
eines Loader Plugins. Hierzu teilten wir die Arbeit auf und konzentrierten uns zunächst
auf das Laden der UH Objekte. Um dies zu bewerkstelligen war eine weitere intensive Beschäftigung
mit dem FIFE Editor Code und dem UH Kartenformat nötig, um weitere Implementierungsdetails
klarzustellen.

Nachdem die Objekte im Editor geladen werden konnten, präsentierten wir unseren Fortschritt
in einem der wöchentlichen Meetings von UH.

Da zu dieser Zeit gerade größere Umbauarbeiten an der Architektur des Spiels vorgenommen wurden,
kam es zu einem Konflikt mit unserem Editor-Code. Änderungen am Spiel zwangen uns, einen Teil
des Codes neu zu schreiben, um mit den neuen Features des Spiels kompatibel zu bleiben. Dies
hätte allerdings nicht vermieden werden können, da die Änderungen bereits vor dem Beginn des
Editor-Projekts geplant waren und nicht verschoben werden konnten. Da wir bereits frühzeitig
auf einem weiter fortgeschrittenen Branch des Git-Repositories gearbeitet hatten, war der
Umfang der nötigen Änderungen noch angemessen.

Im Anschluss daran machten wir uns an die Arbeit, den MapLoader zu bauen. Da die Menge an
Code im Loader Plugin daraufhin relativ stark anwuchs, entschlossen wir uns, den ObjectLoader
vom MapLoader zu trennen und beide mit einem eigenen Plugin auszustatten. Wenig später konnten
wir die Arbeit am MapLoader weitestgehend fertigstellen. Dieser konnte nun die wichtigsten Teile
einer UH-Karte laden.

Da als nächstes der Saver anstand, mussten wir uns nun wieder FIFE zuwenden. Hier war es nötig,
die Plugin Infrastruktur wie oben beschrieben zu erweitern.
% TODO Moritz, hier musst du noch mehr schreiben, da weiß ich zu wenig drüber

Im Anschluss hieran bauten wir den MapSaver.
% TODO Moritz, hier musst du noch mehr schreiben, da weiß ich zu wenig drüber

Ein großer Punkt war jedoch zu diesem Zeitpunkt noch offen: Die Rotation von Objekten auf der Karte.
Bisher lud der MapLoader die Objekte noch ohne Rotation, sie zeigten folglich alle in die gleiche
Richtung. Auch der MapSaver speicherte keine Rotationsdaten. Nach einiger Recherche und ein paar
erfolglosen Versuchen, die Rotation der Gebäude richtig zu setzen, wandten wir uns an die UH-Entwickler.
Wie sich herausstellte war der Code-Abschnitt, der im Spiel die Gebäude rotiert relativ alt und
niemand konnte sich mehr erinnern, wie er funktioniert. Es gab auch keine Dokumentation zu diesem
Punkt.

Daraufhin entschlossen wir uns, das Problem in einer Pair-Programming-Sitzung anzugehen. Dies hatte
den Vorteil, dass wir das Problem gleichzeitig und mit zwei unterschiedlichen Sichtweisen betrachten
konnten, während wir zusammen aktiv an einer Lösung arbeiteten. Hierdurch gelang es uns, eine funktionierende
Rotation der Objekte im MapLoader zu implementieren.

Über die Weihnachtszeit erfolgte dann ein weiterer großer Umbau der Architektur von UH. Das gesamte
Design aller auf der Karte platzierten Objekte wurde umgestellt, woraufhin unser Edior-Code zunächst
nicht mehr funktionstüchtig war. Nach einem vollen Tag an Arbeit waren die Plugins jedoch wieder
einsatzfähig.

%TODO Moritz, hier müsste was über den Island saver stehen

- Island saver
- Fixes vor merge
- Merge
